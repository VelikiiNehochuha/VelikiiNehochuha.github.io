.. title: SICP 1.34 Процедуры как возвращаемые значения, упражнения 1.40-1.46
.. slug: sicp-134-protsedury-kak-vozvrashchaemye-znacheniia-uprazhneniia-140-146
.. date: 2020-01-15 19:17:40 UTC+03:00
.. tags: sicp, scheme, procedures_as_returned_values
.. category: 
.. link: 
.. description: 
.. type: text

\chapter{Упражнение 1.40}

Определите процедуру cubic, которая может быть использована вместе с newtons-method для нахождения корней $x^3+a\cdot x^2+b\cdot x+c$


\begin{codelisting}{scheme}
(define (cubic a b c)
  (lambda (x)
    (+ (* x x x) (* a x x) (* b x) c)))

(newtons-method (cubic 1 1 1) 1)
\end{codelisting}

\chapter{Упражнение 1.41}

Определите процедуру double которая принимает процедуру как аргумент и возвращает процедуру, которая применяет переданную процедуру дважды. К примеру (double inc) должена вернуть процедуру которая два раза делает inc. Какое значение вернет (((double (double double)) inc) 5)?


\begin{codelisting}{scheme}
(define (double f)
  (lambda (x)
    (f (f x))))
(((double (double double)) inc) 0) ; 16
\end{codelisting}

\chapter{Упражнение 1.42}

Пусть f и g две функции одного аргумента. Композицией f после g называется $x\mapsto f(g(x))$. Определите процедуру compose, которая составляет композицию двух функций.

\begin{codelisting}{scheme}
(define (compose f g)
  (lambda (x) (f (g x))))
((compose square inc) 6) ; 49
\end{codelisting}

\chapter{Упражнение 1.43}

Если f числовая функция и n положительное целое, тогда мы можем n раз применить функцию к результату выполнения исходной функции, то есть $f(f(...(f(x))...))$. К примеру, если f(x) = x + 1, тогда если мы применим функцию n раз, то получим функцию f(x)=x+n. Если f - возведение в квадрат,тогда результирующая функция будет x в степени 2n. Напишите процедуру, что бы её можно было использовать вот таким образом ((repeated square 2) 5), используйте compose из упражнения 1.42.

\begin{codelisting}{scheme}
(define (repeated f n)
  (if (= n 1)
      (lambda (x) (f x))
      (compose f (repeated f (- n 1)))))

((repeated square 2) 5) ;; 625
\end{codelisting}

\chapter{Упражнение 1.44}

Идея сглаживающей функции важна при обработке сигналов. Если f - функция и dx - некоторое малое приращение, тогда сглаженная версия функции f, эта функция  значение которой в точке x среднее между f(x-dx) и f(x+dx). Напишете процедуру smooth которая принимает на вход функцию и возвращает сглаженную функцию f. Иногда полезно применить сглаживание несколько раз (то есть сгладить сглаженную функцию) что-бы получить n-раз сглаженную функцию. Покажите как генерировать n раз сглаженную функцию при помощи процедуры repeat из 1.43.

\begin{codelisting}{scheme}
(define dx 0.00001)
(define (smooth f)
  (lambda (x) (average (f (- x dx)) (f (+ x dx)))))

(define (smooth-n f n)
  ((repeated smooth n) f))
((smooth square) 2)
((smooth-n square 5) 2)
\end{codelisting}


\chapter{Упражнение 1.45}

В разделе 1.3.3 мы видели, что вычилсить квадратные корни путем лобового нахождения фиксированной точки от $y\mapsto x/y$ не получается, это удалось починить методом усреднения. Тот же метод работает для нахождения кубического корня. К сожалению процесс не срабатывает для корня четвертой степени. Одного усреднения не достаточно что бы выполнить поиск фиксированной точки для схождения $y\mapsto x/y^3$. C другой стороны если применить усреднение два раза , поиск фиксированной точки начинает сходиться. Поэксперементируйте с количеством усреднений для вычисления корня n-ой степени, то есть на отображении $y\mapsto x/y^{n-1}$. Используя это запишите процедуру для вычисления корня n-ой степени, через fixed-point, average-damp и repeated.

\begin{codelisting}{scheme}
;; 1 ; 1 0
;; 2 ; 2 1
     ; 3 1
;; 4 ; 4 2
     ; 5 2
     ; 6 2
     ; 7 2
;; 8 ; 8 3
     ; 15 3
     ; 16 4
     ; 32 4
     ; 33 5
(define (nth-root x n)
  (define (get-ad-count n i)
    (if (< n (fast-expt 2 i))
        (- i 1)
        (get-ad-count n (+ i 1))))
  (display (get-ad-count n 1))
  (fixed-point ((repeated average-damp (get-ad-count n 1)) (lambda (y) (/ x (fast-expt y (- n 1)))))
               1.0))
(nth-root 27 64)
\end{codelisting}

\chapter{Упражнение 1.46}

Некоторые из численных методов, описанных в этой главе являются примерами примерами чрезвычайно общей вычислитетельной стратегии известной как итеративное улучшение. Итеративное вычисление говорит о том, что для вычисления чего либо, мы начнем с первоначальной догадки для ответа, проверим, является ли догадка достаточно хорошей, иначе улчучшим догадку и продолжим процесс используя улучшенную догадку в качестве новой догадки. Напишите процедуру iterative-improve которая принимает две процедуры в качестве аргументов: метод определяющий достаточно ли хороша догадка и метод улучшения догадки. Iterative-improve должна возвращать процедуру которая принимает в качестве аргумента догадку и улучшает её пока она не проходит проверку на предел. Перепишете процедуру квадратного корня и процедуру fixed-point в терминах iterative-improve.

\begin{codelisting}{scheme}
(define tolerance 0.00001)
(define (close-enough? v1 v2)
  (< (abs (- v1 v2)) tolerance))

(define (iterative-improve good-enough? improve)
  (define (try guess x)
    (let ((next (improve guess)))
      (if (close-enough? guess next)
          next
          (try next x))))
  (lambda (x guess) (try guess x)))

(define (sqrt x)
  (define (improve guess)
    (define (average x y)
      (/ (+ x y) 2))
    (average guess (/ x guess)))
  ((iterative-improve close-enough? improve) x 1.0))

(sqrt 2) ;Value: 1.4142156862745097


(define (fixed-point f)
  ((iterative-improve close-enough? f) 1.0 1.0))
(fixed-point cos) ;Value: .7390822985224024
\end{codelisting}